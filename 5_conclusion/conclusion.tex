本研究ではヒトの運動制御に関して有力な筋シナジー仮説と平衡点仮説に着想を得て,主動筋と拮抗筋の協調を表す筋拮抗比と筋拮抗和を導入し,ニューロサイエンスの知見に基づいた,
ヒト跳躍時の運動機能単位分解を行った.
15秒間の繰返し跳躍を被験者Aと被験者Bに行ってもらい,得られた床反力,運動学データ,EMGを解析した.
共通点と差異についてそれぞれ示す.

まずは共通点として,
\begin{enumerate}
 \item 各筋活動,各関節剛性を離地に向け高めることで跳躍する.
 \item 似通った関節平衡点の推移をしている.
 \item 共通の筋シナジーを用いて制御を行っている. 
\end{enumerate}
そして差異として,
\begin{enumerate}
 \item 接地時の衝撃吸収戦略が異なる.
 \item 被験者Bは股関節,膝関節の剛性を離地時に最大にしている.
 \item シナジースコアが大きく異なり,Aは平衡点を離地前の瞬間に膝を縮める方向にシフトしてしまっている.
\end{enumerate}

特に差異の2,3は跳躍高を決定する要因として考えられる.

今後の課題としては,
異なる衝撃吸収戦略による各関節への負担,エネルギー効率等の差の調査,
差異2,3を踏まえた助言による跳躍高の変化,
被験者を増やしてデータ再現性の確認,
があげられる.

