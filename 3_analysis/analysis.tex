\section{ストレッチセンサ計測アルゴリズム}
\ref{sec:RC回路}にて示す通り、ストレッチセンサの静電容量変化の計測にはRC回路を用いた時定数の計測で行った。
なお、この時定数の計測にはNucleoF303K8といったマイコン評価ボードを使用した。また、それらのコードの記述にはmbedライブラリを使用した。
使用したソースコードは下記リンクにて公開している。計測アルゴリズムはFig.\ref{fig:algorithm}にて示すように
1000Hzで計測を実行し、100Hzごとに計測データの平均値をSerial通信を用いてPCに出力した。
また、2000Hzで出力ピンの状態を切り替え、ストレッチセンサに電荷が残らないようにした。

ソースコード:https://os.mbed.com/users/HidetoN/code/Cap-Sensor/

\begin{figure}[h]
    \begin{center}
     \includegraphics[width=0.9\columnwidth,clip]{./3_analysis/algorithm.eps}
     \caption{計測プログラムにおけるアルゴリズム}
     \label{fig:algorithm}
    \end{center}
\end{figure}

\section{足関節ロボット駆動システム}
足関節ロボットの駆動システムには従来から存在するペダリングロボット、2足歩行ロボットのシステムと同様のシステムを利用した。
以下にそのシステムに関して記述する。

%TODO:システムブロック図入れられると入れたい...
まず、Fig.\ref{fig:compressor}にて示す、エアーコンプレッサーで8気圧まで圧縮空気を作成し、気圧調整弁を用いて6気圧まで減圧を行った。
続いて、制御用PCに搭載されたDAボードより、空気圧制御盤にアナログ信号を送信する。なお、この電圧指令は550段階で制御が行うことが出来る。
この電圧指令を元に空気圧制御盤に搭載された空気圧制御器によって可変的に空気圧の制御を各人工筋に対して行う。

\begin{figure}[h]
    \begin{center}
     \includegraphics[width=0.65\columnwidth,clip]{./3_analysis/compressor.eps}
     \caption{使用したエアーコンプレッサー}
     \label{fig:compressor}
    \end{center}
\end{figure}

\begin{figure}[h]
    \begin{center}
     \includegraphics[width=0.65\columnwidth,clip]{./3_analysis/PC.eps}
     \caption{制御PCとエアー制御盤}
     \label{fig:PC}
    \end{center}
\end{figure}

\section{ストレッチセンサ計測データ処理}
ストレッチセンサの計測データはNucleoF303K8よりSerial通信によって921600bpsで出力される。
制御アルゴリズムの部分でも示したが、1000Hz周期で取得したデータの平均値が100Hz周期に出力される。
このデータをTeraTermのログ取得機能を用いてcsvファイルとして保存した。
なお、足関節ロボット制御用PCの処理の都合上、別PCを用いて結果の取得を行った。
