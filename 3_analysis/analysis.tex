\section{実験方法}
\subsection{伸縮センサ製作}
\begin{enumerate}
    \item 3Dプリンターを用いて伸縮センサのサイズに合った型を用意する.
    \item 導電性布を型に合わせて鋏を用いて切る.
    \item 導電性布に錫めっき線を縫い通し,型の上に導電性布を置く.この際,錫めっき線が型の外に出てくるようにする.
    \item 上から硬化剤を混ぜたシリコン流し込み固まるまで数時間放置.
    \item シリコンが固まったら2枚目の導電性布に1枚目と同様に錫めっき線を通し硬化したシリコンの上に置く.
    \item その上から硬化剤を混ぜたシリコンを薄く塗り固まるまで放置.
    \item 最後のシリコンが硬化したら型から取り外し,錫めっき線に銅線接続し,コネクタをつけて完成.
\end{enumerate}
\subsection{足関節ロボット製作}
今回、足関節ロボットを製作するために

成人男性の半分の質量となるように調整を行った.また重心に関しては,成人男性と同等になるように調整を行った.
\subsection{伸縮センサ計測}
\subsection{データ処理}

\section{結果}