\section{結果}
\subsection{ストレッチセンサ計測精度}
表にまとめると以下の様になった。
\begin{table}[htb]
    \caption{ストレッチセンサ長さ計測結果(mm)}
    \begin{tabular}{|c|c|c|c|c|c|c|c|c|c|c|c|c|c|c|}\hline
        筋肉種類 & 1 & 2 & 3 & 4 & 5 & 6 & 7 & 8 & 9 & 10\\ \hline
        前脛骨筋 & 206 & 208.6 & 211.45 & 245 & 247 & 243 & 239 & 236 & 211.5 & 207.8 \\ \hline
        腓骨筋 & 212.2 & 210.7 & 201.6 & 194.4 & 192.2 & 193.6 & 193.8 & 194.2 & 199.4 & 210.9\\ \hline
        ヒラメ筋 & 235 & 233 & 218 & 209 & 207.3 & 211.8 & 216 & 218 & 223 & 234 \\ \hline
    \end{tabular}
\end{table}
\begin{table}[htb]
    \caption{ストレッチセンサ静電容量計測結果(us)}
    \begin{tabular}{|c|}\hline
        筋肉種類 \\ \hline
        前脛骨筋 \\ \hline
        腓骨筋 \\ \hline
        ヒラメ筋 \\ \hline
    \end{tabular}
\end{table}
\section{考察}
\subsection{ストレッチセンサ計測精度}