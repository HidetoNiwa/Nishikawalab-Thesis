\section{床反力}
フォースプレートにより得られた鉛直方向の床反力データ{\bf Fig. \ref{force}}を用いて周期の切り出しを行う.接地時から次の接地時までを1周期とする.そして接地時から離地時までを立脚期,離地時から接地時までを遊脚期とする.
 
 また,被験者A,Bの1周期における立脚期の割合を{\bf Table }\ref{stance}に示す.このデータによると,AはBに比べ立脚期の割合が大きいことがわかる.
 
\begin{table}[!h]
 \caption{Ratio of Stance.}
 \begin{center}
  \begin{tabular}{|c|c|c|} \hline
    & Subject A & Subject B \\ \hline \hline
  Stance Ratio[%] & 0.616885 & 0.576601 \\ \hline
  \end{tabular}
  \label{stance}
 \end{center}
\end{table}

さらに床反力データを1周期で平均化し({\bf Fig. \ref{FR}}),そのデータから鉛直方向への力積を導出した.そして力積をそれぞれの被験者の体重で除することで,立脚期の速度変化量を導出した{\bf Table }\ref{ISV}.
 
 \begin{table}[!h]
 \caption{Impluse and speed Variation.}
 \begin{center}
  \begin{tabular}{|c|c|c|} \hline
    & Subject A & Subject B \\ \hline \hline
   Impulse[$kg*m/s$] & 783.038 & 899.41 \\ \hline
  Speed Variation[$m/s^2$] & 13.0506 & 14.2763 \\ \hline
  \end{tabular}
  \label{ISV}
 \end{center}
\end{table}

{\bf Table }\ref{ISV}より,被験者Aに比べ被験者Bの方が大きな加速度を得ているため,より高く跳躍していると示される.よって,鉛直方向への跳躍は被験者Bが優れているといえる.

\newpage
\section{運動学データ}
OptiTrackより所得した運動学データを解析し,被験者A,Bの鉛直方向の重心位置の遷移と,各関節角度の遷移を{\bf Fig. \ref{gp}},{\bf Fig. \ref{angle}}に示す.重心位置は簡易化のために両腰関節の中点としている.
 
 重心位置の遷移を見ると,被験者A,B共に同様の軌跡をとっている.離地時の重心位置は被験者Aの$0.647816$,被験者Bの$0.598195$に対し,最高到達点は被験者A$0.913786$,被験者B$0.904918$となっている.この点からも,被験者Bは被験者Aよりも15%高く跳躍していることがわかる.
 また角度に関しては,傾きの大きさこそ違うが,被験者A,B共に相似している.
 
 \newpage

\section{%MVC,筋拮抗和,筋拮抗比}
本実験より,被験者A,Bの右脚から得られた主要8筋($m_1$~$m_8$)の%MVC,4対の筋拮抗和($s_1$~$s_4$),筋拮抗比($r_1$~$r_4$)の時間変化の様子を{\bf Fig. \ref{m14}}~{\bf Fig. \ref{r14}}に示す.
 
\newpage 
各筋のEMGデータを比較すると,被験者AとBで1周期における筋の活動が大きく異なっている.$m_2$,$m_8$は被験者A,B共に似通った筋活動をしているが,$m_1$,$m_4$,$m_5$は被験者Bに対して被験者Aは着地時に強く筋を緊張させている.
 筋拮抗和にも着目すると,接地時に被験者Aは筋拮抗和を高めているのに対し,被験者Bは高めていない.
これは着地時の衝撃吸収における運動戦略の違いであると考えられる.被験者Aは走行時と同様に各関節の剛性を高めることで次の運動に繋げようとしているのに対し,被験者Bは剛性を落とすことによる着地時の衝撃吸収を目的としていると考えられる.
一方,被験者間の共通点として,跳躍踏切の瞬間,つまり離地時に筋拮抗和を大きくし,関節剛性を高める傾向がみられる.特に$s_1$では着地時の剛性の差を除けば類似した軌跡をとっている.

被験者Bは離地時の瞬間ほぼ同時に$s_1$,$s_2$,$s_3$が最大になっている.タイミングを合わせて股関節,膝関節剛性を高めることで被験者Aよりも効率的に筋活動を床反力に還元している可能性が示唆される.

筋拮抗比に関しては,立脚期,遊脚期ともに被験者A,Bの間で似通った値をとっている.$r_4$の遊脚期において大きな差異が生じているが,これは被験者Aは跳躍期に足首を底屈させようとし,被験者Bは足首を伸展させようとしていることを示している.

%ヒトは下肢を一直線にしながら跳躍をするが,筋拮抗比の値を見ると跳躍の瞬間はどれも0.5に極めて近い値をとっており,この傾向が示唆される結果となっている.

\newpage
 \section{シナジーベクトル,シナジースコア} 
筋拮抗比,筋拮抗和より,抽出されたシナジーベクトル動径方向$\ve{u}_R$,偏角方向$\ve{u}_{\it \Phi}$,及びシナジースコアの時間推移を{\bf Fig. \ref{sv}},{\bf Fig. \ref{sc}}にそれぞれ示す.

\newpage

シナジーベクトルは被験者A,B間で非常に類似している.これより,被験者A,B間で個々の筋活動は大きく異なるにも関わらず,筋シナジーは共通していたといえる.

一方で足先平衡点に相当すると考えられるシナジースコアは被験者A,B間で大きく異なっていることが示されている.$W_R$が減少すると脚を伸ばす方向に平衡点を移動させ,増加すると脚を縮める方向に平衡点を移動させる.被験者Aは被験者Bと比べ,着地瞬間に脚を伸ばす方向に平衡点を移動させようとしいる点と,離地する前に伸ばし方向への平衡点の移動を止めてしまっている点が異なっている.前者は衝撃吸収時の戦略の違いであり,後者については,被験者Bは離地瞬間まで平衡点を伸ばし方向に移動させているので,被験者Bに比べ被験者Aが跳躍力に劣る要因の一つであると推測できる.


 