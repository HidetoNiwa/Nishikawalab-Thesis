\section{研究背景}

\section{力学解析に基づく筋協調解析}

\subsection{筋骨格モデルの静力学}

\subsection{筋拮抗比・筋拮抗和の導入とシナジーベクトル}

\section{実験方法}

\subsection{運動計測}

\subsection{データ処理}

\subsection{足関節の平衡点軌道と剛性}

\section{結果}
%
\clearpage
\section{考察}

\subsection{筋シナジー}

\subsection{推定された平衡点軌道と剛性の妥当性}

\subsubsection{トルクと力の推定}

%
\clearpage
\subsubsection{仕事率の推定}
\clearpage
\subsection{サブムーブメント}
\subsubsection{サブムーブメントの抽出}
\subsubsection{抽出されたサブムーブメントの物理的意味}
\section{結言}

\small
\begin{thebibliography}{99}
%%%%%%%%%%%%%%%%%%%%%%%%%%%%%%%%%%%%%%%%%%%%%%%%%%%%%%%%%%%%%%%%%%%%%%%%%%%%%%%
\bibitem{Bernstein1967} N. Bernstein: {\it The co-ordination and regulation of movements}, Oxford, Pergamon, 1967.
\bibitem{Torres-Oviedo2007} G. Torres-Oviedo and L. H. Ting: ``Muscle synergies characterizing human postural responses,'' {\it Journal of neurophysiol.}, vol. 98, no. 4, pp. 2144-2156, 2007.
\bibitem{Cappellini2006} G. Cappellini, Y. P. Ivanenko, R. E. Poppele, and F. Lacquaniti: ``Motor Patterns in Human Walking and Running,'' {\it J. Neurophysiol.}, vol. 95, no. 6, pp. 3426-3437, 2006.
\bibitem{d'Avella2005} A. d'Avella and E. Bizzi: ``Shared and specific muscle synergies in natural motor behaviors,'' {\it Proc. of the National Academy of Sciences of the United States of America}, vol. 102, no. 8, pp. 3076-3081, 2005.
\bibitem{Ting2005} L. H. Ting and J. M. Macpherson: A limited set of muscle synergies for force control during a postural task, {\it Journal of neurophysiol.}, vol. 93, no. 1, pp. 609-613, 2005.
\bibitem{Ivanenko2004} Y. P. Ivanenko, R. E. Poppele, and F. Lacquaniti: Five basic muscle activation patterns account for muscle activity during human locomotion, {\it J. Physiol.}, vol. 556, no. 1, pp. 267-282, 2004.
\bibitem{Ivanenko2006} Y. P. Ivanenko, R. E. Poppele, and F. Lacquaniti: Motor control programs and walking, {\it The Neuroscientist}, vol. 12, no. 4, pp. 339-348, 2006.
\bibitem{Bizzi2008} E. Bizzi, V. C. K. Cheung, A. d'Avella, P. Saltiel, and M. Tresch: Combining modules for movement, {\it Brain Research Reviews}, vol. 57, no. 1, pp. 125-133, 2008.
\bibitem{Feldman2008} A. G. Feldman and M. F. Levin: ``The Equilibrium-Point Hypothesis-Past, Present and Future,'' {\it PROGRESS IN MOTOR CONTROL A Multidisciplinary Perspective}, pp. 699-726, 2008.
\bibitem{Hirai2010} H. Hirai, K. Matsui, T. Iimura, K. Mitsumori, and F. Miyazaki: Modular Control of Limb Kinematics During Human Walking, {\it Proc. of 2010 IEEE Int. Conf. Biomedical Robotics and Biomechatoronics (BioRob2010)}, pp. 716-721, 2010.
\bibitem{Iimura2011} T. Iimura, K. Inoue, H. T. T. Pham, H. Hirai, and F. Miyazaki: Decomposition of Limb Movement based on MuscularCoordination during Human Running, {\it J. Adv. Comp. Intel. \& Intel. Info.}, vol. 15, no. 8, pp. 980-987, 2011.
\bibitem{Inoue2012} K. Inoue, T. Iimura, T. Oku, H. T. T. Pham, H. Hirai, and F. Miyazaki: An Experimental Study of Muscle Coordination and Function during Human Locomotion, {\it BIO web of Conf. /the Int. Conf. SKILLS 2011}, vol. 1, pp. 00040-1-0040-4, 2011.
\bibitem{Ariga2013} 有賀陽平,前田大輔,Hang T. T. Pham,中山かんな,植村充典,平井宏明,宮崎文夫: 筋拮抗比と筋活性度を用いた拮抗駆動装置の線形制御と筋電インタフェースへの応用,日本ロボット学会誌,vol. 31, no. 5,pp. 71-80, 2013.
\bibitem{Uno2014} 宇野かんな, 奥貴紀, 古場啓太郎, 植村充典, 平井宏明, 宮崎文夫: ``水平面内におけるヒト上肢運動時のEMG信号を利用した筋シナジー,平衡軌道および手先剛性の新しい評価法の提案,'' 日本ロボット学会誌, vol. 32, no. 7-8, 2014.
\bibitem{Mitsuda1996} 満田隆, 丸典明, 冨士川和延, 宮崎文夫: ``視空間を用いた逆運動学の線形近似,'' 日本ロボット学会誌, vol. 14, no. 8, pp. 1145-1151, 1996.

\bibitem{Neumann2002} D. A. Neumann, {\it Kinesiology of the Musculoskeletal System}, Mosby, 2002.
\bibitem{Crams2008} E. Criswell: {\it Cram's Introduction to Surface Electromyography}, Second Edition, Jones \& Bartlett Pub, 2008.
\bibitem{Hislop2008} H. Hislop, J. Montgomery: 新・徒手筋力検査法 原著第8版,協同医書出版社, 2008.
%\bibitem{Ariga2012} Y. Ariga, H. T. T. Pham, M. Uemura, H. Hirai and F. Miyazaki: Novel Equilibrium-Point Control of Agonist-Antagonist System with Pneumatic Artificial Muscles, {\it Proc. of the 2012 IEEE Int. Conf. on Robotics and Automation (ICRA2012)}, 1470/1475 (2012)
%\bibitem{Ariga2012_2} Y. Ariga, D. Maeda, H. T. T. Pham, M. Uemura, H. Hirai and F. Miyazaki: Novel Equilibrium-Point Control of Agonist-Antagonist System with Pneumatic Artificial Muscles: II. Application to EMG-based Human-machine Interface for an Elbow-joint System, {\it Proc. of the 2012 IEEE/RSJ Int. Conf. on Intelligent Robots and Systems (IROS2012)}, (2012)
%\bibitem{Inoue2011} K. Inoue, T. Iimura, T. Oku, H. T. T. Pham, H. Hirai, and F. Miyazaki: An Experimental Study of Muscle Coordination and Function during Human Locomotion, {\it BIO web of Conf. /the Int. Conf. SKILLS 2011}, {\bf 1}, 00040-1/0040-4 (2011)
\bibitem{Novacheck1998} T. F. Novacheck: ``The biomechanics of running,'' {\it Gait \& posture}, vol. 7, no. 1, pp. 77-95, 1998.
\bibitem{Eng1995} J. J. Eng and D. A. Winter: ``Kinetic analysis of the lower limbs during walking: what information can be gained from a three-dimensional model?,'' {\it Journal of biomechanics}, vol. 28, no. 6 pp. 753-758, 1995.
\bibitem{Woodworth1899} R. S. Woodworth: ``Accuracy of voluntary movement,'' {\it The Psychological Review: Monograph Supplements}, vol. 3, no. 3, 1899.
\bibitem{Thoroughman2000} K. A. Thoroughman and R. Shadmehr: ``Learning of action through adaptive combination of motor primitives,'' {\it Nature} vol. 407, no. 6805, pp. 742-747, 2000.
\bibitem{Schmidt1988} R. A. Schmidt and T. Lee: {\it Motor Control and Learning, 5E.}, Human kinetics, 1988.

%%%%%%%%%%%%%%%%%%%%%%%%%%%%%%%%%%%%%%%%%%%%%%%%%%%%%%%%%%%%%%%%%%%%%%%%%%%%%%%
\end{thebibliography}
\normalsize
